\documentclass[letterpaper]{resume}
\usepackage{comment}
\usepackage[utf8]{inputenc}
\usepackage{hyperref}
\setcounter{errorcontextlines}{100}
%\usepackage{paralist}

% Multi-colored rules! (John Q is desperate)
\definecolor{ruleendcolor}{rgb}{0.2, 0.2, 0.6}

\begin{document}

\author{Arseniy Yurievich Zaostrovnykh}
\date{\today}
\email{arseniy.zaostrovnykh@epfl.ch}
%\phone{+00000000000}
\maketitle

\section{Research interests}
My research is focused on verification. I try to combine formal methods with symbolic execution in order to reason about packet-processing software.

\section{Education}

\affiliation[PhD in Computer Science]{École Polytechnique Fédérale de Lausanne, Switzerland - \href{http://phd.epfl.ch/edic}{EPFL}, Switzerland}{2014-present}
\begin{itemize}
  \item Advisors: \begin{tabular}{l}
    Prof. George Candea (\href{http://dslab.epfl.ch}{Dependable Systems Lab}),\\
    Prof. Katerina Argyraki (\href{http://nal.epfl.ch}{Network Architecture Lab})
  \end{tabular}
\end{itemize}

\affiliation[MS of Applied Physics and Mathematics]{Moscow Institute of Physics and Technology - \href{http://mipt.ru/en/}{MIPT}, Russia}{2012-2014}
\begin{itemize}
    \item The Department of Radio Engineering and Cybernetics
    \item Specialized in \emph{microprocessor design} and \emph{parallelizing compilers}
    \item Graduation with honors
\end{itemize}

\affiliation[BS of Applied Physics and Mathematics]{Moscow Institute of Physics and Technology - MIPT, Russia}{2008-2012}
\begin{itemize}
    \item The Department of Radio Engineering and Cybernetics\\ The main modules include General Physics,
      Theoretical Physics, Mathematics, Computational Mathematics,
      Computer Science, Operating Systems, Object Oriented Programming, Information Security, etc.
    \item \emph{GPA}: {\bf 5%.00 ?? does precision not matter anymore?
    } out of {\bf 5} (top 1\%)
\end{itemize}
\affiliation{Robotics summer school in Imperial College of London, sponsored by Skolkovo}{2012}
\affiliation{A NatCracker laboratory \url{http://ncedu.ru/moscow/mipt}(rus)}{2010-2011}

\subsection{Successfully taken Stanford on-line courses}
\begin{itemize}
    \item Machine Learning (Andrew Ng)
    \item Algorithms Design and Analysis Part I (Tim Roughgarden)
    \item Compilers (Alex Aiken)
\end{itemize}

\pagebreak
\section{Work and practical experience}

\affiliation{École Polytechnique Fédérale de Lausanne(EPFL), Switzerland}{2014-present}
\begin{itemize}
  \item Research towards software dataplane verification.
\end{itemize}
\affiliation{Google, California, USA}{2015}
\begin{itemize}
  \item Development event tracking for network routing calculation profiling.
\end{itemize}
\affiliation{Samsung Research Institute, Moscow, Russia}{2014}
\begin{itemize}
  \item Developed an AOT compiler of ECMAScript(JS) subset.
\end{itemize}
\affiliation{Intel Corporation, Moscow, Russia}{2010-2013}
\begin{itemize}
  \item Developed a preformance optimizing binary translator from x86 to a new fine-grained parallel architecture.
\end{itemize}
\affiliation{OJS Co. ``Institute of Electronic Control Computers'', Moscow, Russia}{2012}
\begin{itemize}
  \item Developed a technology for components of a base model of prosthesis, controllable by brain impulses.
\end{itemize}
\affiliation{MIPT-Intel Laboratory, Moscow, Russia}{2009-2013}
\begin{itemize}
    \item Used QT to develop an open source graph-visualizer\\ \url{http://code.google.com/p/mipt-vis/}
    \item Mentored a student group in the development of an open source Scheme compiler\\ \url{http://code.google.com/p/mipt-scheme-compiler/}
\end{itemize}
\affiliation{A NetCracker laboratory, Moscow, Russia}{2010-2011}
\begin{itemize}
    \item Used GWT to develop an open source music-listening web service \\\url{https://github.com/necto/natty}
\end{itemize}

\section{Publications}

\begin{itemize}
\item \textbf{A formally verified NAT} {ACM SIGCOMM, 2017}\\
  \emph{Arseniy Zaostrovnykh}, Solal Pirelli, Luis Pedrosa, Katerina Argyraki, George Candea
\item \textbf{Automated synthesis of adversarial workloads for network
    functions} {ACM SIGCOMM, 2018}\\
  Luis Pedrosa, Rishabh Iyer, \emph{Arseniy Zaostrovnykh}, Jonas Fietz, Katerina Argyraki
\item \textbf{A formally verified NAT stack} {ACM SIGCOMM, KBNets workshop, 2018}
  \underline{best paper} \\
  Solal Pirelli, \emph{Arseniy Zaostrovnykh}, George Candea

\item \textbf{Performance contracts for software network functions} {USENIX NSDI, 2019} \\
  Rishabh Iyer, Luis Pedrosa, \emph{Arseniy Zaostrovnykh}, Solal Pirelli, Katerina Argyraki, George Candea
\end{itemize}

\section{Skills Profile}
\subsection{Development skills}
\begin{itemize}
    \item Understand formal methods, computer networks, computer architecture, operating systems, algorithm effectiveness, computational geometry, object-oriented and functional programming.
    \item Have background in classic compiler optimizations.
    \item Experience in a project with millions lines of source code.
    \item Used the following tools, languages and technologies: C++ (2 years of
      industrial, 10 years of academic experience) and C (3 years of industrial
      experience, 4 years of academic experience), CoQ, VeriFast, QT, Java SE/EE(1 year of academic exp.)/ME, XML, \LaTeX, x86/AVR ASM, Git, Subversion, GWT, Maven 2, Bash, Scheme, Common Lisp, SLIME, OpenGL, OpenMP, MPI, AJAX, CSS, SQL (Postgres/Oracle), NoSQL(MongoDB).
    \item Linux (Ubuntu, Arch) and Android operating systems.
\end{itemize}

\subsection{Academic background}
\begin{itemize}
  \item Taught a compilers course for sophomores (2013-2014).
  \item First and second prizes in regional math and physics competitions (2007-2008).
\end{itemize}

\subsection{Languages}
\begin{itemize}
    \item Russian - native
    \item English - fluent (IELTS 7.0, TOEFL 96)
    \item French - advanced
    \item German - basic
\end{itemize}

\flushright{\emph{Last updated: \today}}

\emph{References are available upon request} 



\end{document}
%http://people.epfl.ch/arseniy.zaostrovnykh
